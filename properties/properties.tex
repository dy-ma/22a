%========================%
%        Preamble        %
%========================%
\documentclass[12pt]{article}

%========================%
%        Packages        %
%========================%

\usepackage[utf8]{inputenc}
\usepackage{amsmath}    % Included in amsart package
\usepackage{amsthm}     % 
\usepackage{amssymb}      % 
\usepackage{mathtools}      % Paired Limiter Macros
% \usepackage{mdframed}       % boxes for theorem
\usepackage{enumitem}     % Continuous numbering of lists
\usepackage[hidelinks]{hyperref}

%========================% 
%          Title         %
%========================% 
\title{Linear Algebra Properties}
\author{Dylan Ang}
\date{\today}

%========================% 
%        Theorems        %
%========================% 
\theoremstyle{definition}
\newtheorem{theorem}{Theorem}  % Boxed theorems
\newtheorem{definition}{Definition} % Definitions
\newtheorem{example}{Example}       %
\newtheorem{algorithm}{Algorithm}
\newtheorem*{proof*}{Proof}         % non-numbered
\newtheorem*{remark}{Remark}        %
\numberwithin{equation}{theorem}    % Local equation numbering

\setcounter{tocdepth}{3}      % Show subsubsections in contents

%========================% 
%        Macros          %
%========================% 
\DeclarePairedDelimiter\abs{\lvert}{\rvert}  % Vertical bars
\DeclarePairedDelimiter\norm{\lVert}{\rVert} % Double vertical bars
\newcommand{\drawvec}[1]{                    % matrices on one line
    \begin{bmatrix}
        #1
    \end{bmatrix}
}

%========================% 
%         Document       %
%========================% 
\begin{document}

\maketitle

\tableofcontents

\section{Linear Equations}

\subsection{Systems of Linear Equations}

\begin{definition}
    A system of linear equations has
    \begin{enumerate}
        \item no solution, or
        \item exactly one solution, or
        \item infinitely many solutions.
    \end{enumerate}
    A system of linear equations is said to be \textbf{consistent} if it has either one solution of infinitely many solutions; a system is \textbf{inconsistent} if it has no solution.
    \begin{remark}
        There are two fundamental questions about any linear system.
        \begin{enumerate}
            \item Is the system consistent; that is, does at least one solution \emph{exist?}
            \item If a solution exists, is it the \emph{only} one; that is, is the solution \emph{unique?}
        \end{enumerate}
    \end{remark}
\end{definition}

\begin{definition}
    There are 3 elementary row operations.
    \begin{enumerate}
        \item (Replacement) Replace one row by the sum of itself and a multiple of another row.
        \item (Interchange) Interchange two rows.
        \item (Scaling) Multiply all entries in a row by a nonzero constant.
    \end{enumerate}
    Two matrices are called \textbf{row equivalent} if there is a sequence of elementary row operations that transforms one matrix into the other.
    \begin{remark}
        Row operations are \textbf{reversible}.
    \end{remark}
\end{definition}

\begin{theorem}(Lay 7)
    If the augmented matrices of two linear systems are row equivalent, then the two systems have the same solution set.
\end{theorem}

\subsection{Row Reduction and Echelon Forms}

\begin{definition}
    A rectangular matrix is in \textbf{echelon form} (or \textbf{row echelon form}) if it has the following three properties:
    \begin{enumerate}
        \item All nonzero rows are above any rows of all zeros.
        \item Each leading entry of a row is in a column to the right of the leading entry of the row above it.
        \item All entries in a column below a leading entry are zeros.
    \end{enumerate}
    If a matrix in echelon form satisfies the following additional conditions, then it is in \textbf{reduced echelon form} (or \textbf{reduced row echelon form}):
    \begin{enumerate}[resume]
        \item The leading entry in each nonzero row is 1.
        \item Each leading 1 is the only nonzero entry in its column.
    \end{enumerate}
\end{definition}

\begin{theorem}
    Uniqueness of the Reduced Row Echelon Form.
    Each matrix is row equivalent to one and only one reduced echelon matrix.
\end{theorem}

\begin{definition}
    A \textbf{pivot position} in a matrix A is a location in A that corresponds to a leading 1 in the reduced echelon form of A. A \textbf{pivot column} is a column of A that contains a pivot position.
\end{definition}

\begin{algorithm} Row Reduction.
    The algorithm that follows consists of four steps, and it produces a matrix in echelon form. A fifth step produces a matrix in reduced echelon form. We illustrate the algorithm by an example.
    \begin{enumerate}
        \item Begin with the leftmost nonzero column. This is a pivot column. The pivot position is at the top.
        \item Select a nonzero entry in the pivot column as a pivot. If necessary, interchange rows to move this entry into the pivot position.
        \item Use row replacement operations to create zeros in all positions below the pivot.
        \item Cover (or ignore) the row containing the pivot position and cover all rows, if any, above it. Apple steps 1-3 to the submatrix that remains. Repeat the process until there are no more nonzero rows to modify.
        \item Beginning with the rightmost pivot and working upward and to the left, create zeros above each pivot. If a pivot is not 1, make it 1 by a scaling operation.
    \end{enumerate}
    \begin{remark}
        The combination of steps 1-4 is called the \textbf{forward phase} of the row reduction algorithm. Step 5, which produces the unique reduced echelon form, is called the \textbf{backward phase}.
    \end{remark}
\end{algorithm}

\subsection{Solutions of Linear Systems}

Suppose, for example, that the augmented matrix of a linear system has been changed into the equivalent reduced echelon form $$\begin{bmatrix}
        1 & 0 & -5 & 1 \\
        0 & 1 & 1  & 4 \\
        0 & 0 & 0  & 0
    \end{bmatrix}$$

The associated system of equations is
\begin{align} \label{ex:solution-system}
    x_1 \quad -5x_3 & = 1 \\
    x_2 + x_3       & = 4 \\
    0               & = 0
\end{align}

The variables $x_1$ and $x_2$ corresponding to pivot columns in the matrix are called \textbf{leading/basic variables.} The other variable, $x_3$, is called a \textbf{free variable}.

To solve \ref{ex:solution-system}, solve the system for the leading variables. Ignore the third equation; it offers no restriction on the variables.

\begin{align} \label{ex:solution-simp}
    x_1 & = 1+5x_3 \\
    x_2 & = 4-x_3  \\
    x_3 & is free
\end{align}

The statement ``$x_3$ is free'' means that you are free to choose any value for $x_3$. Once that is done, the formulas in \ref{ex:solution-simp} determine the values of $x_1$ and $x_2$. For instance, when $x_3=$, the solution is (1,4,0).

\subsection{Parametric Descriptions of Solution Sets}

Parametric solution sets use free variables as parameters. Solving a system amounts to finding a parametric description of the solution set or determining that the solution set is empty.

Whenever a system is consistent and has free variables, the solution set has many parametric descriptions.

\begin{theorem}(Lay 21) Existence and Uniqueness Theorem.
    A linear system is consistent if and only if the rightmost column of the augmented matrix is not a pivot column -- that is, if and only if an echelon form of the augmented matrix has no row of the form $$\drawvec{0 & \cdots & 0 & b} \text{ with b nonzero}$$ If a linear system is consistent, then the solution set contains either (i) a unique solution, when there are no free variables, or (ii) infinitely many solutions, when there is at least one free variable.
\end{theorem}

\begin{algorithm}(Lay 21) Row Reduction to Solve Linear System.
    \begin{enumerate}
        \item Write the augmented matrix of the system.
        \item Use the row reduction algorithm to obtain the equivalent augmented matrix in echelon form. Decide whether the system is consistent. If there is no solution, stop; otherwise, go to the next step.
        \item Continue row reduction to obtain the reduced echelon form.
        \item Write the system of equations corresponding to the matrix obtained in step 3.
        \item Rewrite each nonzero equation from step 4 so that its one basic variable is expressed in terms of any free variables appearing in the equation.
    \end{enumerate}
\end{algorithm}

\section{Matrix Algebra}

\subsection{Sums and Scalar Multiples}

\begin{theorem}(Lay 93)
    Let A,B, and C be matrices of the same size, and let r and s be scalars.

    \begin{enumerate}
        \item $A+B=B+A$
        \item $(A+B)+C = A+(B+C)$
        \item $A+0=A$
        \item $r(A+B)=rA+rB$
        \item $(r+s)A=rA+sA$
        \item $r(sA)=(rs)A$
    \end{enumerate}
\end{theorem}

\subsection{Properties of Matrix Multiplication}

\begin{theorem}(Lay 97)
    Let A be an m x n matrix, and let B and C have sizes for which the indicated sums and products are defined.

    \begin{enumerate}
        \item $A(BC)=(AB)C$ (associative law of multiplication)
        \item $A(B+C)=AB+AC$ (left distributive law)
        \item $(B+C)A = BA+CA$ (right distributive law)
        \item $r(AB) = (rA)B =A(rB)$ for any scalar r.
        \item $I_m A=A=AI_n$
        \item $A^0=I$
    \end{enumerate}
\end{theorem}

\begin{remark}(Lay 98)
    WARNING
    \begin{enumerate}
        \item In general, $AB \neq BA$.
        \item The cancellation laws do \emph{not} hold for matrix multiplication. That is, if $AB=AC$, then it is \emph{not} true in general that $B=C$.
        \item If a product $AB$ is the zero matrix, you \emph{cannot} conclude in general that either $A=0$ or $B=0$.
    \end{enumerate}
\end{remark}

\subsection{Transpose}

\begin{theorem}(Lay 99)
    Let A and B denote matrices whose sizes are appropriate for the following sums and products.
    \begin{enumerate}
        \item $(A^T)^T=A$
        \item $(A+B)^T=A^T+B^T$
        \item For any scalar r, $(rA)^T=rA^T$
        \item $(AB)^T=B^TA^T$
    \end{enumerate}
    \begin{remark}
        The transpose of a product of matrices equals the product of their transposes in the reverse order.
    \end{remark}
\end{theorem}

\subsection{Inverse of a Matrix}

\begin{definition}
    $A^{-1}A=I$ and $AA^{-1}=I$. A matrix that is \emph{not} invertible is sometimes called a \textbf{singular matrix}, and an invertible matrix is called a \textbf{nonsingular matrix.}
\end{definition}

\begin{theorem}(Lay 103)
    Let $A=\drawvec{a & b \\ c & d}$. If $ad-bc \neq 0$, then A is invertible and $$A^{-1}=\frac{1}{ad-bc} \drawvec{d & -b \\ -c & a}$$ If $ad-bc=0$, then A is not invertible.
\end{theorem}

\begin{theorem}(Lay 104)
    If A is an invertible n x n matrix, then for each b in $\mathbb{R}^n$, the equation $Ax=b$ has the unique solution $x=A^{-1}b$.
\end{theorem}

\begin{theorem}(Lay 105)
    \begin{enumerate}
        \item If A is an invertible matrix, then $A^{-1}$ is invertible and $$(A^{-1})^{-1}=A$$
        \item If A and B are n x n matrices, then so is AB, and the inverse of AB is the product of the inverses of A and B in the reverse order. That is, $$(AB)^{-1}=B^{-1}A^{-1}$$
        \item If A is an invertible matrix, then so is $A^T$, and the inverse of $A^T$ is the transpose of $A^{-1}$. That is, $$(A^T)^{-1}=(A^{-1})^T$$
    \end{enumerate}
    \begin{remark}
        The product of n x n invertible matrices is invertible, and the inverse is the product of their inverses in the reverse order.
    \end{remark}
\end{theorem}

\subsubsection{Elementary Matrices}

\begin{definition}
    An elementary matrix is one that is obtained by performing a single elementary row operation on an identity matrix. The next example illustrates the three kinds of elementary matrices.
\end{definition}

\begin{example}
    let $$E_1=\begin{bmatrix}
            1  & 0 & 0 \\
            0  & 1 & 0 \\
            -4 & 0 & 1
        \end{bmatrix},
        E_2=\begin{bmatrix}
            0 & 1 & 0 \\
            1 & 0 & 0 \\
            0 & 0 & 1
        \end{bmatrix},
        E_3=\begin{bmatrix}
            1 & 0 & 0 \\
            0 & 1 & 0 \\
            0 & 0 & 5
        \end{bmatrix},
        A=\begin{bmatrix}
            a & b & c \\
            d & e & f \\
            g & h & i
        \end{bmatrix}$$

    Verify that $E_1A=\begin{bmatrix}
            a    & b    & c    \\
            d    & e    & f    \\
            g-4a & h-4b & i-4c
        \end{bmatrix},
        E_2A=\begin{bmatrix}
            d & e & f \\
            a & b & c \\
            g & h & i
        \end{bmatrix},\\
        E_3A=\begin{bmatrix}
            a  & b  & c  \\
            d  & e  & f  \\
            5g & 5h & 5i
        \end{bmatrix}$
\end{example}

\begin{remark}
    If an elementary row operation is performed on an m x n matrix A, the resulting matrix can be written as EA, where the m x m matrix E is created by performing the same row operation of $I_m$.
\end{remark}

\begin{remark}
    Each elementary matrix E is invertible. The inverse of E is the elementary matrix of the same type that transforms E back into I.
\end{remark}

\begin{theorem}(Lay 107)
    An n x n matrix A is invertible if and only if A is row equivalent to $I_n$, and in this case, any sequence of elementary row operations that reduces A to $I_n$ also transforms $I_n$ into $A^{-1}$.
\end{theorem}

\subsubsection{Algorithm for Finding the Inverse of a Matrix}

Row reduce the augmented matrix $\drawvec{A & I}$.  If A is row equivalent to I, then $\drawvec{A & I}$ is row equivalent to $\drawvec{I & A^{-1}}$. Otherwise, A does not have an inverse. (Lay 108)

\subsection{Characterizations of Invertible Matrices}

\begin{theorem}(Lay 112)
    The Invertible Matrix Theorem

    Let A be a square n x n matrix. Then the following statements are equivalent. That is, for a given A, the statements are either all true or all false.
    \begin{enumerate}
        \item A is an invertible matrix.
        \item A is row equivalent to the n x n identity matrix.
        \item A has n pivot positions.
        \item The equation $Ax=0$ has only the trivial solution.
        \item The columns of A form a linearly independent set.
        \item The linear transformations $x \mapsto Ax$ is one-to-one.
        \item The equation $Ax=b$ has at least one solution for each b in $\mathbb{R}^n$.
        \item The columns of A span $\mathbb{R}^n$.
        \item The linear transformation $x \mapsto Ax$ maps $\mathbb{R}^n$ onto $\mathbb{R}^n$.
        \item There is an n x n matrix C such that $CA=I$.
        \item There is an n x n matrix D such that $AD=I$.
        \item $A^T$ is an invertible matrix.
    \end{enumerate}
\end{theorem}

\begin{theorem}
    A linear transformation $T:\mathbb{R}^n \to \mathbb{R}^n$ is said to be invertible if there exists a function $S:\mathbb{R}^n \to \mathbb{R}^n$ such that
    \begin{equation} \label{st invertible}
        S(T(x))= x \text{ for all x in } \mathbb{R}^n
    \end{equation}
    \begin{equation} \label{ts invertible}
        T(S(x))= x \text{ for all x in } \mathbb{R}^n
    \end{equation}
\end{theorem}

\begin{theorem}(Lay 114)
    Let $T:\mathbb{R}^n \to \mathbb{R}^n$ be a linear transformation and let A be the standard matrix for T. Then T is invertible if and only if A is an invertible matrix. In that case, the linear transformation S given by $S(x)=A^{-1}x$ is the unique function satisfying equations \ref{st invertible} and \ref{ts invertible}.
\end{theorem}

\subsection{Matrix Factorizations}

\begin{definition}(Lay 124) The LU Factorization.
    \begin{equation*}
        A=
        \underbrace{
            \begin{bmatrix}
                1 & 0 & 0 & 0 \\
                * & 1 & 0 & 0 \\
                * & * & 1 & 0 \\
                * & * & * & 1 \\
            \end{bmatrix}}_{L}
        \underbrace{
            \begin{bmatrix}
                \blacksquare & *            & * & *            & * \\
                0            & \blacksquare & * & *            & * \\
                0            & 0            & 0 & \blacksquare & * \\
                0            & 0            & 0 & 0            & 0
            \end{bmatrix}}_{U}
    \end{equation*}
    At first, assume that A is an m x n matrix that can be row reduced to echelon form, without row interchanges. (Later, we will treat the general case.) Then A can be written in the form A D LU , where L is an m x m lower triangular matrix with 1’s on the diagonal and U is an m x n echelon form of A. For instance, see the matrix A. Such a factorization is called an LU factorization of A. The matrix L is invertible and is called a unit lower triangular matrix.
    When $A=LU$, the equation $Ax=b$ can be written as $L(Ux)=b$. Writing $y$ for $Ux$, we can find $x$ by solving the pair of equations.
    \begin{align*}
        Ly & =b \\
        Ux & =y
    \end{align*}
\end{definition}

\begin{algorithm} (Lay 126) Algorithm for an LU Factorization
    \begin{enumerate}
        \item Reduce A to an echelon form U by a sequence of row replacement operations, if possible.
        \item Place entries in L such that the same sequence of row operations reduced L to I.
    \end{enumerate}
\end{algorithm}

\section{Vector Spaces}

\begin{definition}
    A \textbf{vector space} is a nonempty set V of objects, called vectors, on which are defined two operations, called \emph{addition} and \emph{multiplication by scalars} (real numbers), subject to the ten axioms (or rules) listed below. The axioms must hold for all vectors u, v, and w in V and for all scalars c and d.
    \begin{enumerate}
        \item The sum of u and v, denoted by $u+v$, is in V.
        \item $u+v=v+u$.
        \item $(u+v)+w=u+(v+w)$.
        \item There is a zero vector 0 in V such that $u+0=u$.
        \item For each u in V, there is a vector $-u$ in V such that $u+(-u)=0$.
        \item The scalar multiple of u by c, denoted by cu, is in V.
        \item $c(u+v)=cu+cv$.
        \item $(c+d)u=cu+du$.
        \item $c(du)=(cd)u$.
        \item $Iu=u$.
    \end{enumerate}
    Using only these axioms, one can show that the zero vector in Axiom 4 is unique, and the vector $-u$, called the negative of u, in Axiom 5 is unique is unique for each u in V.
    \begin{remark}
        For each u in V and scalar c,
        \begin{align}
            0u & = 0     \\
            c0 & = 0     \\
            -u & = (-1)u
        \end{align}
    \end{remark}
\end{definition}

\begin{definition}
    A \textbf{subspace} of a vector space V is a subset H of V that has three properties:
    \begin{enumerate}
        \item The zero vector of V is in H.
        \item H is closed under vector addition. That is, for each u and v in H, the sum $u+v$ is in H.
        \item H is closed under multiplication by scalars. That is, for each u in H and each scalar c, the vector $cu$ is in H.
    \end{enumerate}
\end{definition}

\begin{theorem}
    If $v_1 \cdots v_p$ are in a vector space V, then $Span\{v_1, \cdots, v_p\}$ is subspace of V.
\end{theorem}

\subsection{Null Space}

\begin{theorem}
    The null space of an m x n matrix A is a subspace of $\mathbb{R}^n$. Equivalently, the set of all solutions to a system $Ax=0$ of m homogenous linear equations in n unknowns is a subspace of $\mathbb{R}^n$.
\end{theorem}

\subsubsection{The Column Space of a Matrix}

\begin{definition}
    The \textbf{column space} of a m x n matrix A, written as $Col A$, is the set of all linear combinations of the columns of A. If $A=\{a_1, \cdots, a_n\}$, then $$Col A=Span\{a_1, \cdots, a_n\}$$
\end{definition}

\begin{theorem}
    The column space of an m x n matrix A is a subspace of $\mathbb{R}^m$
\end{theorem}

\begin{remark}
    The column space of an m x n matrix A is all of $\mathbb{R}^m$ if and only if the equation $Ax=b$ has a solution for each b in $\mathbb{R}^m$
\end{remark}

\subsubsection{Contrast between Nul A and Col A}

\begin{enumerate}
    \item [\textbf{Nul A}]
    \item Nul A is a subspace of $\mathbb{R}^n$
    \item Nul A is implicitly defined; that is, you are given only a condition $(Ax=0)$ that vectors in Nul A must satisfy.
    \item It takes time to find vectors in Nul A. Row operations on $\drawvec{A & 0}$ are required.
    \item There is no obvious relation between Nul A and the entries in A.
    \item A typical vector v in Nul A has the property that $Av=0$.
    \item Given a specific vector v, it is easy to tell if v is in Nul A. Just compute Av.
    \item $Nul A=\{0\}$ if and only if the equation $Ax=0$ has only the trivial solution.
    \item $Nul A=\{0\}$ if and only if the linear transformation $x\mapsto Ax$ is one-to-one.
\end{enumerate}

\begin{enumerate}
    \item [\textbf{Col A}]
    \item Col A is a subspace of $\mathbb{R}^m$.
    \item Col A is explicitly defined; that is, you are told how to build vectors in Col A.
    \item It is easy to find vectors in Col A. The columns of A are displayed; others are formed from them.
    \item There is an obvious relation between Col A and the entries in A, since each column of A is in Col A. 
    \item A typical vector v in Col A has the property that the equation $Ax=v$ is consistent.
    \item Given a specific vector v, it may take time to tell if v is in Col A. Row operations on $\drawvec{A & v}$ are required.
    \item $Col A=\mathbb{R}^m$ if and only if the equation $Ax=b$ has a solution for every b in $\mathbb{R}^m$.
    \item $Col A=\mathbb{R}^m$ if and only if the linear transformation $x \mapsto Ax$ maps $\mathbb{R}^n$ onto $\mathbb{R}^m$.
\end{enumerate}

\begin{definition}
    A \textbf{linear transformation} T from a vector space V into a vector space W is a rule that assigns to each vector x in V is a unique vector $T(x)$ in W, such that
    \begin{align}
        T(u+v) = T(u) + T(v) &\quad \text{for all u, v in V, and} \\
        T(cu) = cT(u) &\quad \text{for all u in V and all scalars c.}
    \end{align}
\end{definition}

\end{document}