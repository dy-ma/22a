%========================%
%        Preamble        %
%========================%
\documentclass[12pt]{amsart}

%========================%
%        Packages        %
%========================%

\usepackage[utf8]{inputenc}
% \usepackage{amsmath}    % Included in amsart package
% \usepackage{amsthm}     % 
% \usepackage{amssymb}    % 
\usepackage{mathtools}      % Paired Limiter Macros
\usepackage{mdframed}       % boxes for theorem
\usepackage[hidelinks]{hyperref}

%========================% 
%          Title         %
%========================% 
\title{Determinants}
\author{Dylan Ang}
\date{\today}

%========================% 
%        Theorems        %
%========================% 
\newmdtheoremenv{theorem}{Theorem}  % Boxed theorems
\newtheorem{definition}{Definition} % Definitions
\newtheorem*{proof*}{Proof}         % non-numbered
\newtheorem*{remark}{Remark}        %
\newtheorem{example}{Example}      %
\numberwithin{equation}{theorem}    % Local equation numbering

%========================% 
%        Macros          %
%========================% 
\DeclarePairedDelimiter\abs{\lvert}{\rvert}  % Vertical bars
\DeclarePairedDelimiter\norm{\lVert}{\rVert} % Double vertical bars
\newcommand{\drawvec}[1]{                    % matrices on one line
    \begin{bmatrix}
        #1
    \end{bmatrix}
}

%========================% 
%         Document       %
%========================% 
\begin{document}

\maketitle

\tableofcontents

\begin{definition}
    The determinant of a square matrix is a single number which carries important info about the matrix. We write det A.
\end{definition}

\begin{equation}
    \text{Let } A =
    \begin{bmatrix}
        a & b \\
        c & d
    \end{bmatrix}
    \text{ Then } det(A)=ad-bc
\end{equation}

$$\text{Recall that }A^{-1}=\frac{1}{ad-bc}
    \begin{bmatrix}
        a & b \\
        c & d
    \end{bmatrix}$$

$$\Rightarrow A \text{ is invertible} \Leftrightarrow det(A) \neq 0$$

$$\text{Can be written } det(A) =
    \begin{vmatrix}
        a & b \\
        c & d
    \end{vmatrix}$$

We can define det(A) for any $n*n$ matrix recursively
\begin{align} \label{formula:determinant}
    \text{For 3 by 3 matrices}           \\
    A      & = \begin{bmatrix}
        a_{1 1} & a_{1 2} & a_{1 3} \\
        a_{2 1} & a_{2 2} & a_{2 3} \\
        a_{3 1} & a_{3 2} & a_{3 3}
    \end{bmatrix} \\
    det(A) & =
    a_{1 1}
    \begin{vmatrix}
        a_{2 2} & a_{2 3} \\
        a_{3 2} & a_{3 3}
    \end{vmatrix}
    - a_{1 2}
    \begin{vmatrix}
        a_{2 1} & a_{2 3} \\
        a_{3 1} & a_{3 2}
    \end{vmatrix}
    + a_{1 3}
    \begin{vmatrix}
        a_{2 1} & a_{2 2} \\
        a_{3 1} & a_{3 2}
    \end{vmatrix}
\end{align}

\section{Rules for Determinant}

\begin{enumerate}
    \item $det(A)$ changes signs when we exchange two rows.
    \item $det(A)$ is unchanged if a multiple of one row is subtracted from another row.
          \begin{itemize}
              \item $\Rightarrow A \text{ and } U \text{ have the same determinant}$.
          \end{itemize}
    \item For a triangular matrix, it's determinant is the product of the diagonal entries.
    \item A matrix with a row of zeros has a $det(A)=0$.
          \begin{itemize}
              \item $\Rightarrow$ if $A$ has two identical rows $\Rightarrow det(A)=0$.
          \end{itemize}
    \item $det(A)=$ product of it's pivots.
    \item $det(A)\neq 0$ if and only if $A$ is invertible.
    \item $det(A) = det(A^{T})$.
\end{enumerate}

\begin{theorem}
    $$det(AB) = det(A)*det(B)$$
    $$A = \begin{bmatrix}
            a & b \\
            c & d
        \end{bmatrix}, B=\begin{bmatrix}
            p & q \\
            r & s
        \end{bmatrix}$$

\end{theorem}

\begin{proof}
    \begin{align*}
        det(A)*det(B) & = (ad-bc) * (ps-qr)                                     \\
                      & = adps - bcps - adqr + bcqr                             \\
                      & = (ad+br)(cq+ds) - (aq+bs)(cp+dr)                       \\
        AB            & =  \begin{bmatrix}
            a & b \\
            c & d
        \end{bmatrix}\begin{bmatrix}
            p & q \\
            r & s
        \end{bmatrix}
        = \begin{bmatrix}
            ap+br & aq+bs \\
            cp*dr & cq+ds
        \end{bmatrix}                                            \\
        det(AB)       & = (ap+br)(cq+ds)-(cp+dr)(aq+bs) \square
    \end{align*}
\end{proof}

\section{Linearity of Determinant}

\begin{equation*}
    \begin{vmatrix}
        xa+yA & xb+yB \\
        c     & d
    \end{vmatrix}
    = x(ad-bc)+y(Ad-Bc) \\
\end{equation*}
$$\Rightarrow \text{det is linear in row 1}$$

\begin{example}
    \begin{equation*}
        A = \begin{bmatrix}
            1  & 2 & 3 \\
            4  & 0 & 1 \\
            -2 & 1 & 1
        \end{bmatrix}
    \end{equation*}
    \begin{align*}
        det(A) & = 1*0*1+4*1*3+(-2)*2*1     \\
               & = (-2*0*3 - 1*1*1 - 4*2*1) \\
               & = 0+12-4-(0-3-8)           \\
               & = 8+11 = 19
    \end{align*}
\end{example}

\begin{example}\textbf{Identity.}
    \begin{equation*}
        det \begin{bmatrix}
            1 & 0 & 0 \\
            0 & 1 & 0 \\
            0 & 0 & 1
        \end{bmatrix} = 1
    \end{equation*}
\end{example}

\begin{example}\textbf{Row Exchanges.}
    By Rule 1 for determinants, two row exchanges $\rightarrow$ determinant flips sign
    \begin{align*}
        det \begin{bmatrix}
            0 & 0 & 1 \\
            0 & 1 & 0 \\
            1 & 0 & 0
        \end{bmatrix} & = -1        \\
        det \begin{bmatrix}
            0 & 0 & 0 & 1 \\
            0 & 0 & 1 & 0 \\
            0 & 1 & 0 & 0 \\
            1 & 0 & 0 & 0
        \end{bmatrix} & = -(-1) = 1
    \end{align*}
    This happens because the $det(I^{4}) = 1$
\end{example}

\section{Example: Finding Determinant}

\begin{example}\textbf{Finding Determinant.}
    Compute \begin{align*}
        det \begin{bmatrix}
            1  & 0 & 0 & 0 \\
            2  & 1 & 0 & 0 \\
            -3 & 2 & 0 & 0 \\
            4  & 4 & 1 & 1
        \end{bmatrix} & = \Leftarrow \text{lower-triangular matrix} \\
                                       & = 1*1*0*1 = 1
    \end{align*}

    \paragraph{Why?}
    \begin{itemize}
        \item Lower-triangular matrices are invertible if and only if diagonal entries are not 0.
        \item Therefore, this matrix is not invertible.a
        \item determinant of 0 means non-invertible.
        \item Therefore the determinant is 0.
    \end{itemize}
\end{example}

\begin{theorem}
    If $Q$ is an orthogonal square matrix, then $det(Q)=\pm 1$
    Prove with $Q^{T}Q=I$, $Q^{T}=Q^{-1}$, and $det(A)det(A^{-1}) = det(AA^{-1})$
\end{theorem}

\section{Parallelogram Theorem}

\begin{theorem}
    \textbf{Parallelogram}
    Let P be the parallelogram spanned by the rows of $A$. Then
    \begin{equation}
        vol(P) = \lfloor det(A) \rfloor
    \end{equation}
\end{theorem}

\begin{example}
    \begin{align*}
        A                            & = \begin{bmatrix}
            4 & 1 \\
            2 & 3
        \end{bmatrix}  \\
        \lfloor det(A) \rfloor       & = \lfloor 4(3) - 1(2) \rfloor \\
        \text{area of parallelogram} & = 12 - 2 = 10
    \end{align*}
\end{example}

\section{With LU Factorization}

\begin{equation}
    det(AA) = det(LU) = det(L)*det(U)
\end{equation}

\begin{example}
    \begin{align*}
        A      & = \begin{bmatrix}
            4 & 2 & 0 \\
            2 & 4 & 2 \\
            0 & 2 & 4
        \end{bmatrix}       \\
        A      & = LU                               \\
        L      & = \begin{bmatrix}
            1           & 0           & 0 \\
            \frac{1}{2} & 1           & 0 \\
            0           & \frac{2}{3} & 1
        \end{bmatrix}       \\
        U      & = \begin{bmatrix}
            4 & 2 & 0           \\
            0 & 3 & 2           \\
            0 & 0 & \frac{8}{3}
        \end{bmatrix}       \\
        det(A) & = det(L)det(U)                     \\
               & = (1*1*1) * (4*3*\frac{8}{3}) = 32
    \end{align*}
\end{example}

\end{document}