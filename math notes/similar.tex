%========================%
%        Preamble        %
%========================%
\documentclass[12pt]{amsart}

%========================%
%        Packages        %
%========================%

\usepackage[utf8]{inputenc}
% \usepackage{amsmath}    % Included in amsart package
% \usepackage{amsthm}     % 
% \usepackage{amssymb}    % 
\usepackage{mathtools}      % Paired Limiter Macros
\usepackage{mdframed}       % boxes for theorem
\usepackage[hidelinks]{hyperref}

%========================% 
%          Title         %
%========================% 
\title{Similar Matrices}
\author{Dylan Ang}
\date{\today}

%========================% 
%        Theorems        %
%========================% 
\newmdtheoremenv{theorem}{Theorem}  % Boxed theorems
\newtheorem{definition}{Definition} % Definitions
\newtheorem*{proof*}{Proof}         % non-numbered
\newtheorem*{remark}{Remark}        %
\newtheorem*{example}{Example}      %
\numberwithin{equation}{theorem}    % Local equation numbering

%========================% 
%        Macros          %
%========================% 
\DeclarePairedDelimiter\abs{\lvert}{\rvert}  % Vertical bars
\DeclarePairedDelimiter\norm{\lVert}{\rVert} % Double vertical bars
\newcommand{\drawvec}[1]{                    % matrices on one line
    \begin{bmatrix}
        #1
    \end{bmatrix}
}

%========================% 
%         Document       %
%========================% 
\begin{document}

\maketitle

\begin{definition}
    We call two matrices A,B similar if there exists an invertible matrix C such that
    $A=C^{-1}BC$ (or $B=CAC^{-1}$)
\end{definition}

\begin{theorem}
    Similar matrices have the same eigenvalues.
\end{theorem}

\begin{proof}
    Assume A,B are similar $\rightarrow A=C^{-1}BC$. Consider $A\vec{x}=\lambda \vec{x}$. Then \begin{align*}
         & \underbrace{(C^{-1}BC)}_A \vec{x} = \lambda \vec{x}                                    \\
         & \underbrace{CC^{-1}}_I BC\vec{x} = C \lambda x                                         \\
         & B \underbrace{C\vec{x}}_{\vec{v}} = \lambda C\vec{x} \text{, denote } \vec{v}=C\vec{x} \\
         & B \vec{v} = \lambda \vec{v}                                                            \\
         & \Rightarrow \lambda \text{ is an eigenvalue of B with eigenvector } \vec{v}(=C\vec{x})
    \end{align*}
\end{proof}

\begin{theorem}
    Suppose X diagonalizes both A and B. Then $AB=BA$ (i.e., A and B commute).
\end{theorem}

\begin{proof}
    \begin{align*}
         & X^{-1} AX=\Lambda_1 , X^{-1}BX=\Lambda_2
        \quad ,\text{(A,B have the same eigenvectors)}                                                                             \\
         & \Rightarrow A = X\Lambda_1X^{-1} , B=X\Lambda_2 X^{-1}                                                                  \\
         & AB = X \Lambda_1 \underbrace{X^{-1} X}_I\Lambda_2 X^{-1}                                                                \\
         & = X\Lambda_1 \Lambda_2 X^{-1}                                                                                           \\
         & = X\Lambda_2 \Lambda_1 X^{-1} \quad ,(\Lambda_1 \Lambda_2 = \Lambda_2 \Lambda_1)\text{ since diagonal matrices commute} \\
         & = \underbrace{X \Lambda_2}_B \underbrace{X^{-1}X}_I \underbrace{Lambda_1 X^{-1}}_A
    \end{align*}
\end{proof}

\begin{example}
    \begin{align*}
         & A = \drawvec{4                                           & 1 \\1&4}, B = \drawvec{3&1\\1&3} \\
         & \text{eigenvectors of A:} \frac{1}{\sqrt{2}} \drawvec{-1     \\1}, \frac{1}{\sqrt{2}} \drawvec{1\\1} \\
         & \text{eigenvectors of B:}                                    \\
         & \quad det(B-\lambda I) = 0                                   \\
         & \quad det \drawvec{3-\lambda  1                              \\ 1 & 3-\lambda I} = 0 \Rightarrow \cdots \mu_1=2, \mu_2=4 \\
         & \Rightarrow \text{B has same eigenvalues as A}               \\
         & \Rightarrow \text{B,A commute}                               \\
         & AB =
        \begin{bmatrix}
            4 & 1 \\
            1 & 4
        \end{bmatrix}
        \begin{bmatrix}
            3 & 1 \\
            1 & 3
        \end{bmatrix}
        = \begin{bmatrix}
            13 & 7  \\
            7  & 13
        \end{bmatrix}                                    \\
         & BA =
        \begin{bmatrix}
            3 & 1 \\
            1 & 3
        \end{bmatrix}
        \begin{bmatrix}
            4 & 1 \\
            1 & 4
        \end{bmatrix}
        \begin{bmatrix}
            13 & 7  \\
            7  & 13
        \end{bmatrix}
    \end{align*}
\end{example}

\end{document}