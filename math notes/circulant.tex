%========================%
%        Preamble        %
%========================%
\documentclass[12pt]{amsart}

%========================%
%        Packages        %
%========================%

\usepackage[utf8]{inputenc}
% \usepackage{amsmath}    % Included in amsart package
% \usepackage{amsthm}     % 
% \usepackage{amssymb}    % 
\usepackage{mathtools}      % Paired Limiter Macros
\usepackage{mdframed}       % boxes for theorem
\usepackage[hidelinks]{hyperref}

%========================% 
%          Title         %
%========================% 
\title{Circulant Matrices}
\author{Dylan Ang}
\date{\today}

%========================% 
%        Theorems        %
%========================% 
\newmdtheoremenv{theorem}{Theorem}  % Boxed theorems
\newtheorem{definition}{Definition} % Definitions
\newtheorem*{proof*}{Proof}         % non-numbered
\newtheorem*{remark}{Remark}        %
\newtheorem*{example}{Example}      %
\numberwithin{equation}{theorem}    % Local equation numbering

%========================% 
%        Macros          %
%========================% 
\DeclarePairedDelimiter\abs{\lvert}{\rvert}  % Vertical bars
\DeclarePairedDelimiter\norm{\lVert}{\rVert} % Double vertical bars
\newcommand{\drawvec}[1]{                    % matrices on one line
    \begin{bmatrix}
        #1
    \end{bmatrix}
}

%========================% 
%         Document       %
%========================% 
\begin{document}

\maketitle

\begin{definition}
    An nxn matrix C is called circulant if C =
    \begin{align*}
        \begin{bmatrix}
            c_1     & c_n     & c_{n-1} & \cdots     & c_2     \\
            c_2     & c_1     & c_n     & \cdots     & \cdots  \\
            c_3     & c_2     & c_1     & \cdots     & \cdots  \\
            \cdots  & c_3     & c_2     & \cdots     & c_{n-1} \\
            c_{n-1} & \cdots  & c_3     & \cdots     & c_n     \\
            c_n     & c_{n-1} & c_{n-1} & \cdots c_2 & c_1
        \end{bmatrix}
    \end{align*}
\end{definition}

\begin{example}
    \begin{align*}
        C = \begin{bmatrix}
            4 & 1 & 2 & 3 \\
            3 & 4 & 1 & 2 \\
            2 & 3 & 4 & 1 \\
            1 & 2 & 3 & 4
        \end{bmatrix}
    \end{align*}
\end{example}

\section{Discrete Fourier Transform}

\begin{definition}
    The nxn DFT matrix is given by
    \begin{align*}
        F_n = \frac{1}{\sqrt{n}} \begin{bmatrix}
            1 & 1            & 1               & 1               & \cdots & 1                   \\
            1 & \omega       & \omega^2        & omega^3         & \cdots & \omega^{n-1}        \\
            1 & \omega^2     & \omega^4        & omega^6         & \cdots & \omega^{2(n-1)}     \\
            1 & \omega^3     & \omega^6        & omega^9         & \cdots & \omega^{3(n-1)}     \\
              &              &                 & \cdots          &        &                     \\
            1 & \omega^{n-1} & \omega^{2(n-1)} & \omega^{3(n-1)} & \cdots & \omega^{(n-1)(n-1)}
        \end{bmatrix}
    \end{align*}
    where $\omega = e^{2\pi i / n}$ and $i=\sqrt{-1}$
\end{definition}

\begin{theorem}
    A circulant matrix is diagonalized by the DFT matrix. Then $F_{n}^{-1} C F_n = \Lambda$ where $\Lambda=$ diagonal matrix, containing eigenvalues of C.
    \begin{remark}
        $F_{n}^{-1} = F_{n}^{H}$ and $C^{-1}=F_n \Lambda^{-1}F_{n}^{H}$
    \end{remark}
    Multiplication by the DFT matrix $F_N$ can be done in $n\log{n}$ computations instead of $n^2$ computations. Done via Fast Fourier Transforms.
\end{theorem}

\end{document}