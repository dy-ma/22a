%========================%
%        Preamble        %
%========================%
\documentclass[12pt]{amsart}

%========================%
%        Packages        %
%========================%

\usepackage[utf8]{inputenc}
% \usepackage{amsmath}    % Included in amsart package
% \usepackage{amsthm}     % 
% \usepackage{amssymb}    % 
\usepackage{mathtools}      % Paired Limiter Macros
\usepackage{mdframed}       % boxes for theorem
\usepackage[hidelinks]{hyperref}

%========================% 
%          Title         %
%========================% 
\title{Symmetric Matrices}
\author{Dylan Ang}
\date{\today}

%========================% 
%        Theorems        %
%========================% 
\newmdtheoremenv{theorem}{Theorem}  % Boxed theorems
\newtheorem{definition}{Definition} % Definitions
\newtheorem*{proof*}{Proof}         % non-numbered
\newtheorem*{remark}{Remark}        %
\newtheorem*{example}{Example}      %
\numberwithin{equation}{theorem}    % Local equation numbering

%========================% 
%        Macros          %
%========================% 
\DeclarePairedDelimiter\abs{\lvert}{\rvert}  % Vertical bars
\DeclarePairedDelimiter\norm{\lVert}{\rVert} % Double vertical bars
\newcommand{\drawvec}[1]{                    % matrices on one line
    \begin{bmatrix}
        #1
    \end{bmatrix}
}

%========================% 
%         Document       %
%========================% 
\begin{document}

\maketitle

Recall: S is a symmetric matrix if $S=S^T$

\begin{theorem}
    If S is a symmetric matrix, then all its eigenvalues are real values.
\end{theorem}

\begin{proof}
    Let $S\vec{x} = \lambda \vec{x}$. $\Lambda$ could be complex-valued: $\Lambda=a+ib$, $i=\sqrt{-1}$. We want to show that $\Lambda$ is real-valued, i.e. $b=0$.

    Let $\bar\Lambda=a-ib$ be the conjugate of $\Lambda$ and $\bar{\vec{x}}$ be the conjugate of $\vec{x}$.

    Know: $\bar{\Lambda \vec{x}} = \bar\Lambda \bar\vec{X}$

    Also know: $S=\bar S$, since S is real-valued.

    \begin{align*}
        S\vec{x} = \lambda \vec{x}                            & \text{, Now take conjugate on both sides}      \\
        \bar{S\vec{x}} = \bar{\lambda \vec{x}}                &                                                \\
        \bar S \bar \vec{x} = \bar\lambda \bar\vec{x}         & , \text{S is real valued, so } \bar S = S      \\
        S \bar \vec{x} = \bar\lambda \bar\vec{x}              & , \text{ Now take the transpose on both sides} \\
        (S\bar \vec{x})^{T} = (\bar \lambda \bar \vec{x})^{T} &                                                \\
        \bar\vec{x}^T S^{T} = \bar\vec{x}^{T} \bar \lambda    & , \text{ S is symmetric, so } S=S^{T}          \\
        \bar\vec{x}^T S = \bar\vec{x}^T \bar\lambda           & , \text{ Now take product with } \vec{x}       
    \end{align*}
    \begin{equation}\label{normal x}
        \bar\vec{x}^T Sx = \bar\vec{x}^T \bar\lambda x 
    \end{equation}

    Now consider $S\vec{x}=\lambda\vec{x}$ take dot product with $\vec{x}$
    \begin{equation}\label{vec x}
        \bar\vec{x}^T S \vec{x} = \bar\vec{x}^T \lambda \vec{x} 
    \end{equation}

    Left sides of \eqref{normal x} and \eqref{vec x} are equal $\Rightarrow$ right sides are equal.
    \begin{align*}
        \Rightarrow \bar\vec{x}^T \bar\lambda \vec{x} &= \bar\vec{x}\lambda\vec{x} \\
        \bar\lambda\underbrace{\bar\vec{x}^T\vec{x}} &= \lambda\underbrace{\bar\vec{x}^T\vec{x}} \\
        \text{These terms are equal: } & \bar\vec{x}^T = \sum_{k=1}^{n}{\bar x_k x_k} = \sum_{k=1}^{n}{\abs{x_{k}^{2}}} \\
        & \Rightarrow \bar\vec{x}^T\vec{X} \neq 0 \\
        \Rightarrow \bar\lambda &= \lambda \\
        \Rightarrow a-ib &= a+ib \Rightarrow b=0 \Rightarrow \lambda \text{ is real} 
    \end{align*}
\end{proof}

\end{document}